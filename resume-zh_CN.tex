% !TEX TS-program = xelatex
% !TEX encoding = UTF-8 Unicode
% !Mode:: "TeX:UTF-8"

\documentclass{resume}
\usepackage{zh_CN-Adobefonts_external} % Simplified Chinese Support using external fonts (./fonts/zh_CN-Adobe/)
% \usepackage{NotoSansSC_external}
% \usepackage{NotoSerifCJKsc_external}
% \usepackage{zh_CN-Adobefonts_internal} % Simplified Chinese Support using system fonts
\usepackage{linespacing_fix} % disable extra space before next section
\usepackage{cite}
\usepackage{hyperref}

\begin{document}
\pagenumbering{gobble} % suppress displaying page number

\name{杨韬}

\basicInfo{
  \email{yanggtau@gmail.com} \textperiodcentered\ 
  \phone{(+86) 151-8160-2870} \textperiodcentered\ 
  \github[https://github.com/yangtau]{https://github.com/yangtau}
  %\linkedin[billryan8]{https://www.linkedin.com/in/billryan8}
  }
 
\section{\faGraduationCap\  教育背景}
\datedsubsection{\textbf{电子科技大学}, 成都, 四川}{2017年9月 -- 至今}
\textit{本科}\ 电子信息工程, 计算机科学与技术 
\newline
\textit{GPA:}\ 3.97/4 \  \textit{排名:} 6/205 


\section{\faUsers\ 实习/项目经历}

\begin{onehalfspacing}
\datedsubsection{\textbf{腾讯实习}}{2020年6月 -- 至今}
\role{后台开发}{基础架构部门}
\begin{itemize}
  \item 参与 PB 级数据存储和查询系统重构
  \item 参与基于虚拟机的 JS 混淆器开发
\end{itemize}
\end{onehalfspacing}


\begin{onehalfspacing}
\datedsubsection{\textbf{Hedgehog 编程语言}}{2018年12月 -- 至今}
\role{编译器}{个人项目}
Hedgehog 编程语言, \url{https://github.com/yangtau/hedgehog}
\begin{itemize}
  \item 基于栈式虚拟机
  \item 实现了条件, 循环, 函数等基本功能
  \item 内置字符串, 数组, 字典等类型
  \item 垃圾回收
\end{itemize}
\end{onehalfspacing}


\begin{onehalfspacing}
\datedsubsection{\textbf{MySQL 存储引擎}}{2019 年4月 -- 至今}
\role{数据库}{个人项目}
一个基于 B+ 树的存储引擎, \url{https://github.com/yangtau/example-engine}
\end{onehalfspacing}


\begin{onehalfspacing}
\datedsubsection{\textbf{星辰工作室}}{2017年9月 -- 2018年9月}
\role{Web 开发}{校园论坛}
\begin{itemize}
  \item 参与电子科技大学官方论坛清水河畔的开发维护
  \item 开发论坛安卓应用
\end{itemize}
\end{onehalfspacing}


\begin{onehalfspacing}
\datedsubsection{\textbf{更多项目}}{}
请查看 \url{https://yangtau.me/projects.html}
\end{onehalfspacing}

% Reference Test
%\datedsubsection{\textbf{Paper Title\cite{zaharia2012resilient}}}{May. 2015}
%An xxx optimized for xxx\cite{verma2015large}
%\begin{itemize}
%  \item main contribution
%\end{itemize}

\section{\faCogs\ IT 技能}
% increase linespacing [parsep=0.5ex]
\begin{itemize}[parsep=0.5ex]
  \item 编程语言: C > Dart > Python > C++ > Java
\end{itemize}

\section{\faInfo\ 其他}
% increase linespacing [parsep=0.5ex]
\begin{itemize}[parsep=0.5ex]
  \item 博客: \url{https://yangtau.me}
  \item 语言: 英语 - 六级
\end{itemize}

%% Reference
%\newpage
%\bibliographystyle{IEEETran}
%\bibliography{mycite}
\end{document}
